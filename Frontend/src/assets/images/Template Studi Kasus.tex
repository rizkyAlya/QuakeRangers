\documentclass{article}
\usepackage{amsmath}
\usepackage[a4paper, total={6in, 8in}]{geometry}
\usepackage{enumitem}
\usepackage{graphicx}
\usepackage{float}
\usepackage[parfill]{parskip}
\usepackage{fancyhdr}
\usepackage{lipsum} % Untuk contoh teks
\usepackage{titlesec}

% Mengaktifkan fancyhdr
\pagestyle{fancy}
\fancyhf{} % Menghapus header dan footer default

% Menampilkan judul bab di footer
\fancyfoot[C]{\nouppercase{\leftmark}} % Menampilkan judul chapter di tengah footer
\fancyfoot[R]{\thepage} % Menampilkan nomor halaman di kanan footer

% Garis pemisah untuk header dan footer
\renewcommand{\headrulewidth}{0pt} % Tidak ada garis di header
\renewcommand{\footrulewidth}{0.4pt} % Garis di footer

\begin{document}

\title{\textbf{Evaluasi Progress Report 2 Studi Kasus Kelompok 20} \\ Analisis Perilaku Pengguna Transportasi Umum}
\author{Christopher Satya Fredella Balakosa, Bintang Siahaan, Fabsesya Muhammad P.I]}
\date{}
\maketitle

\section{Chapter 7}
\subsection{Deskripsi}
\textbf{Soal:}

Di sebuah kota besar, bus umum tiba di terminal dengan jadwal acak. Setiap penumpang yang tiba di terminal akan menunggu bus berikutnya tiba. Waktu tunggu bus bagi setiap penumpang disimulasikan sebagai variabel acak eksponensial dengan rata-rata waktu tunggu \( 5 \) menit.

Misalkan \( X_1, X_2, \dots, X_n \) adalah waktu tunggu penumpang yang tiba secara acak, yang mengikuti distribusi eksponensial dengan nilai ekspektasi \( \mu = 5 \) menit. Misalkan \( M_n(X) \) adalah rata-rata sampel dari waktu tunggu \( n \) penumpang yang datang secara acak.

\textbf{Latar Belakang:}

Pemerintah kota ingin merencanakan kapasitas transportasi umum yang efisien berdasarkan estimasi jumlah penumpang pada jam sibuk. Untuk itu, estimasi yang akurat mengenai rata-rata jumlah penumpang dan variansi sangat diperlukan. Proses estimasi ini menggunakan teori statistik untuk menghasilkan prediksi yang dapat diandalkan dalam perencanaan angkutan umum.

\textbf{Permasalahan/Pertanyaan:}

Dari informasi yang diberikan, jawab pertanyaan berikut:
\begin{enumerate}
    \item Tentukan estimasi rata-rata jumlah penumpang dan variansi dari sampel yang diambil.
    \item Hitunglah Mean Square Error (MSE) dari estimasi rata-rata jumlah penumpang.
    \item Apakah estimasi rata-rata jumlah penumpang dapat diterima sebagai estimasi konsisten jika jumlah sampel diperbesar?
\end{enumerate}

\textbf{Parameter Penting:}

Beberapa parameter yang digunakan dalam soal ini adalah:
\begin{itemize}
    \item \( X_1, X_2, \dots, X_n \): waktu tunggu penumpang pada perjalanan ke-1 hingga ke-\( n \).
    \item \( M_n(X) \): rata-rata sampel dari waktu tunggu pada \( n \) perjalanan.
    \item \( \mu \): rata-rata waktu tunggu pada populasi.
    \item \( S_n^2 \): variansi sampel dari waktu tunggu penumpang.
    \item \( \sigma^2 \): variansi populasi waktu tunggu penumpang.
\end{itemize}

\subsection{Model}
\begin{enumerate}
    \item Misalkan \( X_1, X_2, \dots, X_n \) adalah waktu tunggu penumpang pada kedatangan ke-1 hingga ke-\( n \). 
    \item Estimasi rata-rata waktu tunggu pada \( n \) penumpang adalah: 
    \[
    \text M_n(X) = \frac{1}{n} \sum_{i=1}^n X_i
    \]
    \item Rata-rata waktu tunggu dalam populasi adalah $\mu = 5$ menit dan variansi $\sigma^2 = 25$ (karena distribusi eksponensial memiliki variansi $\sigma^2 = \mu^2$).
\end{enumerate}
\subsection{Asumsi}
\begin{enumerate}
    \item Waktu tunggu setiap penumpang $X$ mengikuti distribusi eksponensial dengan rata-rata $\mu = 5$ menit dan variansi $\sigma^2 = 25$.
    \item Estimasi rata-rata waktu tunggu $M_n(X)$ adalah tak bias.
    \item Setiap waktu tunggu penumpang dianggap independen satu sama lain.
\end{enumerate}
\subsection{Solusi}

\textbf{Pembahasan Bagian (a)}

\textbf{Menghitung Jumlah Sampel yang Dibutuhkan:}

Berdasarkan \textbf{Teorema 7.1}, variansi dari rata-rata sampel \( M_n(X) \) adalah:
\[
\text{Var}(M_n(X)) = \frac{\sigma^2}{n} = \frac{25}{n}
\]
Diketahui bahwa \( \sigma^2 = 25 \) dan variansi dari rata-rata sampel \( M_n(X) \) harus lebih kecil atau sama dengan 0.25, maka:
\[
\frac{25}{n} \leq 0.25
\]
Untuk menyelesaikan untuk \( n \), kita kalikan kedua sisi persamaan dengan \( n \) dan kemudian bagi dengan 0.25:
\[
25 \leq 0.25n
\]
Kemudian, menyelesaikan untuk \( n \):
\[
n \geq \frac{25}{0.25} = 100
\]

Jadi, jumlah sampel yang dibutuhkan adalah \( \mathbf{100} \) sampel agar variansi dari rata-rata sampel \( M_n(X) \) tidak lebih dari 0.25.

\textbf{Pembahasan Bagian (b)}

\textbf{Menghitung Mean Square Error (MSE):}

Berdasarkan \textbf{Teorema 7.6}, MSE dari estimasi rata-rata waktu tunggu \( M_n(X) \) adalah:
\[
MSE =\text{Var}(M_n(X)) = \frac{\sigma^2}{n}
\]
Diketahui bahwa \(n = 50)\) dan \( \sigma^2 = 25 \), maka:
\[
MSE = \frac{25}{n} = 0.25
\]
Jadi, MSE dari estimasi rata-rata waktu tunggu untuk 50 sampel adalah 0.5.

\textbf{Pembahasan Bagian (c)}

\textbf{Konsistensi Estimasi dengan Memperbesar Sampel:}

Berdasarkan \textbf{Teorema 7.7}, estimasi rata-rata waktu tunggu \( M_n(X) \) akan menjadi lebih konsisten dengan bertambahnya jumlah sampel. Ini karena variansi estimasi rata-rata sampel berkurang seiring dengan bertambahnya jumlah sampel n.
\[
\text{Var}(M_n(X)) = \frac{\sigma^2}{n}
\]
Dengan \( n = 200 \), variansi estimari rata-rata waktu tunggu adalah:
\[
\text{Var}(M_n(X)) = \frac{25}{200} = 0.125
\]
Semakin besar jumlah sampel n, semakin kecil variansi estimasi, dan estimasi akan semakin mendekati nilai rata-rata sebenarnya dari populasi, yang menunjukkan bahwa estimasi ini \textbf{konsisten}.

\end{document}